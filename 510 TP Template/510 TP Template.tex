% Import the LaTeX style file and load some common packages
\documentclass[final]{siamart1116}
\usepackage{amsfonts}
\usepackage{amsopn}

% Declare the title, author, and any other front matter

% Title
\title{Applications of Mathematics in [Application]}

% Authors: Full names and addresses
\author{Kai Li}

% Page headers (visible after the cover page)
\headers{AMS 510 Term Project}{K. Li}

% Start the document
\begin{document}
\maketitle

\section{Introduction}
\label{sec:intro}
Introduce your application here. Questions you may consider addressing include:
\begin{itemize}
\item What is your application?
\item Why are you interested in this application?
\item What details about the application are necessary for the reader to appreciate the math concepts?
\item What are some problems or issues in the application? You can potentially use these to foreshadow the math concepts in later sessions.
\end{itemize}

You can have equations, if desired, but should not feel required to have equations. In many cases an equation is probably unnecessary; a conceptual discussion is usually sufficient. That said, if you include an equation, such as
\begin{equation}
F(\mathbf{A}) = \left\{ x \cdot \mathbf{A}x : x\in\mathbb{C}^n, \left\| x \right\|=1 \right\},
\label{eq:fov}
\end{equation}
make sure the terms and ideas are introduced. For instance, in Eq.\ \eqref{eq:fov}, $\mathbf{A}\in\mathbb{C}^{n\times n}$ is some square, complex matrix and $F(\mathbf{A})$ is the numerical range of $\mathbf{A}$ \cite{bk:horn}. Equations are parts of sentences, and should be punctuated appropriately.

References are also important. A simple \textsc{Bib}\TeX ``database'' is provided to you that includes entries for our linear algebra textbook (cite key: \texttt{bk:strang}) \cite{bk:strang} and our advanced calculus textbook (cite key: \texttt{bk:petrovic}) \cite{bk:petrovic}. You can (and should) add other resources to this database to include in your bibliography.

I suspect that this section will take most of the first page.

\section{Linear Algebra Math Concept}
\label{sec:la}
Describe and discuss your linear algebra math concept, and its relation to your application, here. Remember to include
\begin{itemize}
\item An explanation of the math concept. A formal definition may be included, but is probably too much detail for the discussion.
\item A description of how the math concept relates to your application.
\end{itemize}
I anticipate that this section will be one or two paragraphs long (roughly 1/3 of a page).

\section{Advanced Calculus Math Concept}
\label{sec:ac}
This section is similar to that on a linear algebra math concept (Section \ref{sec:la}), but will feature a math concept from advanced calculus.

\section{Miscellaneous Math Concept}
\label{sec:misc}
The final section is similar to the previous two, highlighting a math concept from any branch or discipline of mathematics.

The math concepts can be presented in any order (choose one that makes the most sense or that improves the flow of the project) and the sections can be renamed, if desired. Each math concept should have its own section in the document, and please make it clear which math concept is the ``linear algebra'' math concept, which is the ``advanced calculus'' math concept, and which is the third math concept.

% Put references, in BibTeX format, in the file refs.bib
\bibliographystyle{siamplain}
\bibliography{refs}

\end{document}
